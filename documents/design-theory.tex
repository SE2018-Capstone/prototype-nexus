\documentclass[11pt, oneside]{article}   	% use "amsart" instead of "article" for AMSLaTeX format
\usepackage{geometry}                		% See geometry.pdf to learn the layout options. There are lots.
\geometry{letterpaper}                   		% ... or a4paper or a5paper or ... 
\usepackage[parfill]{parskip}
%\geometry{landscape}                		% Activate for for rotated page geometry
%\usepackage[parfill]{parskip}    		% Activate to begin paragraphs with an empty line rather than an indent
\usepackage{graphicx}				% Use pdf, png, jpg, or eps§ with pdflatex; use eps in DVI mode
								% TeX will automatically convert eps --> pdf in pdflatex		
\usepackage{amssymb}

\title{Nexus - Design Theory}
\author{Shiranka Miskin, Dhruv Lal, Sam Maier, Navin Mahabir, Justin Sacbibit}
%\date{}							% Activate to display a given date or no date
\let\endtitlepage\relax
\begin{document}
\maketitle
\section {Brief Description}
Our project aims to create a platform for dynamic and interactive desktop
widgets and themes.  A simple example is a desktop background that displayed the
weather, or a volume slider widget.  The advantage we have in accomplishing this
is the new open source library Electron, which allows cross-platform desktop
applications to be built with web technologies.  We would be able to create user
interfaces very quickly in JavaScript, and would not have to make significant
efforts to be cross-platform, unlike many existing applications.

\section{Design Disciplines}
The needs of our project are entirely related to software.  We will mainly be
using JavaScript, which the majority of our team members are already familiar with
through co-op experiences.  Where we will need to learn a substantial amount is
in interfacing with the native APIs of each operating system.  Our plan to
acquire this knowledge is mainly by going through online resources.  We do not
have any collaborators, however there is an electron community slack channel
where we can priodically ask for assistance.

\section{Creating Alternatives}
A fitting local analogy to our task is the Rainmeter platform for Windows, which
lets users customize their desktops with clickable images and text.  Users
create their own themes using a custom file format that Rainmeter specifies, and
over its many years of existence, there have been a huge number of widgets and
themes people have created.  What we would need to create would have to be at
least as powerful as Rainmeter, which we plan to achieve through the benefits of
using web technologies.

Another possible local analogy is the widgets on Android phones.  There are
widgets to see the weather, view one's calendar, view their notes or todo lists,
or even control applications such as Spotify.

A distant analogy would be that of a command centre.  What comes to mind are
scenes in science fiction movies where people are looking at complicated screens
displaying a number of different stats and buttons.  The goal of our application
is to make common tasks such as checking the weather much closer at hand
compared to having to open a web browser and search it up every day.

\section{Selecting from Alternatives}
The main choice that we had to make was between focusing our efforts on creating
more of a dynamic and interactive desktop background, compared to focusing on
separate widgets that one could bring up perhaps by pressing a hotkey.

The dynamic and interactive desktop background path was our initial goal,
however the tradeoff was that the functionality in Electron required to
implement this only works on Mac and Linux.  To succeed we would need to fork
Electron and implement this functionality ourselves.

The separate widget path would also adhere to our "command centre" ideal and
would avoid the Windows compatibility issue, however after much brainstorming we
could not come up with a well designed product idea around this.

We decided to take the risk of having to implement the Windows functionality we
need in Electron and focus on creating a platform for dynamic and interactive
desktop backgrounds.  It would be more difficult, however it would both be a
learning opportunity and would allow us to create a product which we can be more
proud of.


\section{Prototyping}
The prototypes we built for this SE390 project were exploratory and horizontal
in nature.  Electron was unfamiliar to all but one of our group members, and
there were numerous unknowns about the feasibility of our idea that we had to
confirm.  We tested out creating a native node module, as they will be crucial
in accomplishing operating-system specific native tasks which we cannot do
through JavaScript.  We also tested out implementing multiple simple widgets
which use existing node libraries to familiarize our team members with Electron
and figure out.

\section{Normal and Radical Design}

Our project is a slightly normal design as it follows the same ideas as the
existing Rainmeter platform.  Rainmeter is, however, the only application like
this and is more limited in its functionality. Therefore, there could be more
radical leaps we could make throughout the project.  Our more radical approach
is using Electron, which opens up many more possibilities than what is possible
with Rainmeter.

\section{Laws of Software Design}

All the work we put in for this 390 project has been exploratory, and we plan to
throw it all away as described by Brooks on Prototyping.  Since our plan is for
this to be a platform for others to create their own themes and widgets on top
of, we will have to work towards designing a very simple and intuitive system
for those to be built upon.  To accomplish this, we may take inspiration from
laws such as Einstein's Razor, Saint-Exupery's Razor, and Hoare's Razor.

\section{Understanding Project Risk}

The main risk involved is one of skills.  In order to accomplish certain native
tasks and ensure that our product can be ran on Mac, Linux, and Windows, we will
have to learn how to accomplish all of our native tasks on all three of these
operating systems.  We will also have to learn about the Electron internals and
we are putting our trust in our ability to eventually be able to implement the
functionality we require that is currently missing in Electron.  One way we can
mitigate this risk is by contacting folks with experience in these areas, such
as an Electron or a Rainmeter maintainer.

\end{document}  
